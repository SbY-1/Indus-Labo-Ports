\documentclass[a4paper, 11pt]{article}

\usepackage[left=1.25in, right=1.25in, top=1.0in, bottom=1.0in]{geometry}
\usepackage[latin1]{inputenc}
\usepackage[T1]{fontenc}
\usepackage[francais]{babel}
\usepackage{fontspec}
\usepackage{amscd} 
\usepackage{amsfonts}
\usepackage{color}
\usepackage{fancyhdr}
\usepackage{listings}
\usepackage{tabularx}
 \pagestyle{fancy}

%Entêtes et pieds de page
\fancyhead[C]{Informatique industrielle: Labo 2 Suite}
\fancyhead[L]{Rahali \& Schmidt}
\fancyhead[R]{2014-2015}

\definecolor{mygreen}{rgb}{0,0.6,0}
\definecolor{mygray}{rgb}{0.5,0.5,0.5}
\definecolor{mymauve}{rgb}{0.58,0,0.82}


\lstset{ %
  backgroundcolor=\color{white},   % choose the background color; you must add \usepackage{color} or \usepackage{xcolor}
  basicstyle=\footnotesize,        % the size of the fonts that are used for the code
  breakatwhitespace=false,         % sets if automatic breaks should only happen at whitespace
  breaklines=true,                 % sets automatic line breaking
  captionpos=b,                    % sets the caption-position to bottom
  commentstyle=\color{mygreen},    % comment style
  deletekeywords={...},            % if you want to delete keywords from the given language
  escapeinside={\%*}{*)},          % if you want to add LaTeX within your code
  extendedchars=true,              % lets you use non-ASCII characters; for 8-bits encodings only, does not work with UTF-8
  frame=single,                    % adds a frame around the code
  keepspaces=true,                 % keeps spaces in text, useful for keeping indentation of code (possibly needs columns=flexible)
  keywordstyle=\color{blue},       % keyword style
  language=Octave,                 % the language of the code
  morekeywords={*,...},            % if you want to add more keywords to the set
  numbers=left,                    % where to put the line-numbers; possible values are (none, left, right)
  numbersep=5pt,                   % how far the line-numbers are from the code
  numberstyle=\tiny\color{mygray}, % the style that is used for the line-numbers
  rulecolor=\color{black},         % if not set, the frame-color may be changed on line-breaks within not-black text (e.g. comments (green here))
  showspaces=false,                % show spaces everywhere adding particular underscores; it overrides 'showstringspaces'
  showstringspaces=false,          % underline spaces within strings only
  showtabs=false,                  % show tabs within strings adding particular underscores
  stepnumber=1,                    % the step between two line-numbers. If it's 1, each line will be numbered
  stringstyle=\color{mymauve},     % string literal style
  tabsize=2                       % sets default tabsize to 2 spaces
}

%%%%%%%%%%%%%%%%%%%%%%%%%%%%%%%%%%%%%%%%%%%%%
%							Page de garde						 %
%%%%%%%%%%%%%%%%%%%%%%%%%%%%%%%%%%%%%%%%%%%%%
\makeatletter
\def\clap#1{\hbox to 0pt{\hss #1\hss}}%
\def\ligne#1{%
\hbox to \hsize{%
\vbox{\centering #1}}}%
\def\haut#1#2#3{%
\hbox to \hsize{%
\rlap{\vtop{\raggedright #1}}%
\hss
\clap{\vtop{\centering #2}}%
\hss
\llap{\vtop{\raggedleft #3}}}}%
\def\bas#1#2#3{%
\hbox to \hsize{%
\rlap{\vbox{\raggedright #1}}%
\hss
\clap{\vbox{\centering #2}}%
\hss
\llap{\vbox{\raggedleft #3}}}}%
\def\maketitle{%
\thispagestyle{empty}\vbox to \vsize{%
\haut{}{\@blurb}{}
\vfill
\vspace{1cm}
\begin{flushleft}
\usefont{OT1}{ptm}{m}{n}
\huge \@title
\end{flushleft}
\par
\hrule height 4pt
\par
\begin{flushright}
\usefont{OT1}{phv}{m}{n}
\Large \@author
\par
\end{flushright}
\vspace{1cm}
\vfill
\vfill
\bas{}{\@location, le \@date}{}
}%
\cleardoublepage
}
\def\date#1{\def\@date{#1}}
\def\author#1{\def\@author{#1}}
\def\title#1{\def\@title{#1}}
\def\location#1{\def\@location{#1}}
\def\blurb#1{\def\@blurb{#1}}
\date{\today}
\author{}
\title{}
\location{Liège}\blurb{}
\makeatother
\title{Informatique industrielle: Labo 2 Suite}
\author{Rahali Nassim \& Schmidt Sébastien -  M18}
\location{Liège}
\blurb{%
Haute Ecole de la Province de Liège
}%

\begin{document}
\maketitle

\tableofcontents
\newpage

\section{Organisation du projet}
	Pour la réalisation de ce projet, nous avons essayé au maximum de scinder le programme en parties. Nous aurions pu créer un seul fichier de code source mais la compréhension du code en aurait pâti. C'est la raison pour laquelle une série de processus a été créés afin de rendre le code compréhensible. 
	
	De plus, les processus vont se créer automatiquement, il n'y aura pas besoin de lancer les processus manuellement. On peut considérer le processus Gestion comme le père des processus. Il aura pour tâche de lancer X processus Bateau et Y processus Port. Le port créera ensuite ses propres processus Quai et son processus GenVehicle. Le nombre de processus bateau et Port à lancer sera spécifier dans un fichier de configuration qui sera lu au démarrage de l'application.
	
\section{Structures de données}
	Pour ce projet, nous avons défini deux structures de données: Une pour les bateaux et une pour les quais. Ces structures seront contenues dans des mémoire partagées accessibles par tous les processus.
	\subsection{Structure pour les bateaux} \label{struct_boat}
		\lstinputlisting[language=C, firstline=21, lastline=33]{../lib/Common.h}
		Dans un premier temps, on a défini deux énumération pour la direction ainsi que la position du bateau. Ces énumération seront utilisées dans la structure. Par rapport au point de vue théorique, nous avons rajouté deux données: l'index du bateau ainsi qu'un indicateur pour savoir si les données ont été modifiées ou pas. Cette données est uniquement utilisées pour l'affichage et n'a donc pas grand intérêt. La mémoire partagée aura une taille égale à 6 fois la taille de cette structure.
	\subsection{Structure pour les quais}
		\lstinputlisting[language=C, firstline=35, lastline=39]{../lib/Common.h}
		Chaque port possède X quais, les informations essentielles de ces quais seront contenues dans une mémoire partagée par port. Dans cette structure, on spécifiera l'index du quai ainsi que le bateau qui y est actuellement accosté. Si aucun bateau n'est présent le boat\_index vaudra -1 et sera libre.	

\section{Les différents processus}
	\subsection{Processus Gestion}
		Comme spécifié précédemment, ce processus va créé des processus fils grâce à la fonction fork(). On va ensuite exécuter un fichier grâce a la fonction execl qui peut également passer des arguments à ce processus. Aux processus Bateau, on leur fournira leur numéro d'index: Comme il y a 6 bateaux, chaque processus aura un index compris entre 0 et 5. Ce numéro sera utilisé par le processus Quai et Port afin de savoir quel bateau est sur le point de rentrer dans le port ou d'accoster. Quant au processus Port, on lui fournir le nom de son port (Calais, Dunkerque ou Douvre) afin de notamment savoir le nombre de quais que le port possède mais également pour créer des sémaphores. En effet les ports ont besoin de sémaphores uniques: on va dans ce cas utiliser le nom du port concaténé à un nom de sémaphore commun pour avoir un nom de sémaphore unique et facilement retrouvable pour les autres processus qui en auront besoin (comme le bateau lors de son entrée).
		
			\lstinputlisting[language=C, firstline=43, lastline=75]{../src/Gestion.c}
	\subsection{Processus Bateau}
		Le rôle du bateau dans un premier temps est d'initialiser sa propre structure contenue dans la mémoire partagée des bateaux avec une position au milieu de la mer. On a vu au point \ref{struct_boat} que l'énumération des directions contenait une valeur supplémentaire: UNDEFINED qui sera utilisée pour initialiser la direction du bateau. 
		
		Une fois l'initialisation terminée, on rentre dans la boucle qui va lire l'état du bateau dans la mémoire partagée et va effectuer une série de tâche en fonction de sa position. 
		\lstinputlisting[language=C, firstline=52, lastline=54]{../src/Boat.c}	
		L'index correspondant à l'identifiant du bateau. Il permet de récupérer la structure correspondant à ce dit bateau. Quand les différentes modifications en fonction de la position du bateau, une mise à jour de la structure de la mémoire partagée est effectuée.
		\subsubsection{Au milieu de l'eau}
			\lstinputlisting[language=C, firstline=68, lastline=82]{../src/Boat.c}
			S'il s'agit du premier voyage, on définit une destination aléatoirement. Par contre si ce n'est pas le premier voyage, les bateaux venant de France (Calais ou Dunkerque) doivent impérativement aller vers Douvre. Au contraire s'ils proviennent d'Angleterre, une destination est tiré de manière aléatoire entre Calais et Dunkerque. On effectue ensuite une simulation de la traversée.
		\subsubsection{Entrée dans le port}
			\lstinputlisting[language=C, firstline=94, lastline=108]{../src/Boat.c}
			Lorsque le bateau arrive à l'entrée d'un port, il doit prévenir le port et incrémenté le compteur d'arrivée de ce port. Tout d'abord il doit récupérer le mutex et la mémoire partagée appartenant à ce port. Pour cela on utilise le nom du port qui est stocké dans la structure du bateau et on reconstruit le nom du mutex concerné. Le même travaille est effectué pour la mémoire partagée et le sémaphore permettant d'avertir le port d'une arrivée.	
			
			Le bateau doit ensuite attendre l'autorisation du port pour pouvoir rentrer. Cette synchronisation se fait grâce à un sémaphore débloquer par le port.
		
		\subsubsection{A quai}
			%
			%	TODO
			%
		
		\subsubsection{Sortie du port}
			\lstinputlisting[language=C, firstline=166, lastline=177]{../src/Boat.c}
			La sortie dun port se fait exactement de la même manière que l'entrée à la différence qu'ici on incrémente le compteur de départ du port. Il faut donc récupérer les ressources correspondantes.
	\subsection{Processus Port}
		Les ports commencent par la création des ressources qui leur sont propres et initialisent les compteurs de départ et d'arrivée à zéro. Ils créent également deux ou trois quais en fonction du port qu'ils représentent ainsi que processus permettant de générer un embarquement/débarquement de véhicules. Pour la création des processus fils, on utilise la même méthode que pour la création des processus depuis le processus Gestion: C'est-à-dire une combinaison de fork() et de execl. On passe en paramètres aux processus Quai le nom du port ainsi que l'index du quai. Ces informations seront utiles pour créer des ressources uniques.
		
		Ils se mettent ensuite dans l'attente d'un bateau grâce à une sémaphore initialisée à zéro également: 
		\lstinputlisting[language=C, firstline=45, lastline=47]{../src/Port.c}
		
		Quand le processus Port est débloqué, il commence par traité les départs des bateaux si il y'en a. Il va devoir lire le compteur de départ écrit dans une mémoire partagée. Si des bateaux désirent quitter, il doit décrémenter le compteur et avertir le bateau concerné qu'il peut quitter le port. 
		\lstinputlisting[language=C, firstline=52, lastline=78]{../src/Port.c}
		
		Sinon il aucun bateau n'est prêt à quitter, c'est qu'un bateau est sur le point d'entrer. Il faut donc lui trouver un quai libre auquel il peut venir accoster et il faut ensuite décrémenter le compteur d'arrivée et lui prévenir d'entrer. Quand le quai est réservé, le numéro du bateau est placé dans la structure du quai en question. Le bateau devra allé lire la mémoire partagée du port contenant l'ensemble des quais pour savoir quel quai lui est destiné.
		\lstinputlisting[language=C, firstline=90, lastline=106]{../src/Port.c}
		
		Il ne reste plus qu'à s'occuper des entrées après la réservation du quai et prévenir le bateau qu'il peut entrer.
			
	\subsection{Processus Quai}
		Un processus quai commence par la création des ressources qui lui sont propres : 
		\begin{itemize}
			\item Une sémaphore quai.
			\item Une mutex pour l'accès relatif à la SHM des quais.
			\item Une SHM relative aux quais.
			\item Une sémaphore pour le processus de générations de véhicules.
		\end{itemize}
		Il dispose également d'une variable de type Dock permettant de stocker les informations relatives à lui même depuis la SHM des quais.
		\lstinputlisting[language=C, firstline=28, lastline=34]{../src/Dock.c}
		Il entre ensuite dans une boucle dans laquelle il commence par attendre sur sa sémaphore quai. Celle ci sera signalée par un bateau en temps voulu.
		\lstinputlisting[language=C, firstline=36, lastline=44]{../src/Dock.c}
		Quand un processus Quai est signalé, le bateau est à quai. On peut donc récupérer son index (disponible via la SHM des quais et stockée dans la variable Dock). On peut alors réaliser l'ouverture les Messages Queues de ce bateau.
		\lstinputlisting[language=C, firstline=52, lastline=53]{../src/Dock.c}
		De manière logique, il s'agit alors de réaliser le débarquement. Ceci est réalisé de la manière suivante : on récupère d'abord le nombre de messages dans les Messages Queues, puis une fois ces valeurs obtenues, on consomme les messages des deux Messages Queues (une pour les Camions et l'autre pour les Voitures et Camionnettes).
		\lstinputlisting[language=C, firstline=59, lastline=62]{../src/Dock.c}
		\lstinputlisting[language=C, firstline=66, lastline=85]{../src/Dock.c}
		Avant de repartir sur un nouveau tour de boucle, le processus Quai va signaler la sémaphore relative à la génération de véhicules ce qui va permettre de remplir à nouveau les Messages Queues du bateau avant son départ. Notons aussi que le débarquement réalisé par un processus Quai peut prendre un certain temps (0.25 secondes par véhicules).
		\lstinputlisting[language=C, firstline=86, lastline=89]{../src/Dock.c}
		
		
	\subsection{Processus GenVehicle}
		Il y a une version de ce processus associée à chaque port. Comme le système que nous mettons en place nécessite de la synchronisation, ce processus va aussi devoir utiliser les sémaphores et autres structures présentées précédemment.
		\begin{itemize}
			\item La sémaphore concernant la génération de véhicule (qui lui est propre).
			\item La mutex protégeant l'accès à la SHM des bateaux.
			\item La sémaphore permettant la synchronisation avec le bateau qui a terminé son embarquement (et donc le prévenir qu'il peut partir).
			\item La SHM des bateaux.
		\end{itemize}
		On définit également une variable de type Boat pour stocker le bateau recherché par la fonction \texttt{get\_actual\_boat()} ainsi que les deux descripteurs qui permettront l'accès aux Messages Queues du bateau.
		\lstinputlisting[language=C, firstline=6, lastline=12]{../src/GenVehicle.c}
		Une fois l'ouverture des ressources nécessaires terminées, ce processus entre également dans une boucle. Il commence par attendre sur sa propre sémaphore concernant la génération de véhicule et comme celle-ci est initialisée avec une valeur de 0, le processus est en attente. 
		\lstinputlisting[language=C, firstline=43, lastline=46]{../src/GenVehicle.c}
		Cette sémaphore sera donc signalée par un processus quai associé au même port que le processus \textit{GenVehicle}. Le processus de génération de véhicule peut alors réaliser ce à quoi il est destiné : générer des voitures, des camionnettes et des camions pour le bateau qui est en attente! Il va donc commencer par rechercher de quel bateau il s'agit. 
		\lstinputlisting[language=C, firstline=48, lastline=56]{../src/GenVehicle.c}
		Une fois celui obtenu (et donc l'accès aux Messages Queues), nous utilisons \texttt{srand(getpid())} et \texttt{rand()} pour fournir des nombres aléatoires de camions, voitures et camionnettes.
		\lstinputlisting[language=C, firstline=58, lastline=64]{../src/GenVehicle.c}
		Les constantes sont définies dans le header de \textit{GenVehicle.c}.
		\lstinputlisting[language=C, firstline=16, lastline=18]{../lib/GenVehicle.h}
		Quand ces valeurs sont générées, il reste donc à remplir les Messages Queues ce qui est donc fait par la fonction \texttt{mq\_send()}. Si on prend par exemple le remplissage de la Message Queue représentant les camionnettes et les voitures :
		\lstinputlisting[language=C, firstline=68, lastline=96]{../src/GenVehicle.c}
		On peut constater que les voitures entrent d'abord et que les camionnettes suivent (NB : ici aussi, 0,25s par véhicule pour l'embarquement). On remarquera également les deux priorités diffèrent et sont en fait des constantes définies dans le header de \textit{GenVehicle.c}.
		\lstinputlisting[language=C, firstline=13, lastline=14]{../lib/GenVehicle.h}
		Les camionnettes ayant une priorité plus elevée seront bien celles qui débarqueront en premier dans le quai. La dernière étape est donc de prévenir le bateau qui attend que l'embarquement soit terminé et de lui permettre de prendre son état suivant :
		\lstinputlisting[language=C, firstline=113, lastline=117]{../src/GenVehicle.c}
		Après quoi le processus de génération repart, content de ses prouesses, pour une nouvelle boucle.

\section{Code source}
	\subsection{Config.cfg}
		\lstinputlisting[language=C]{../Config.cfg}
	\subsection{Common.h}
		\lstinputlisting[language=C]{../lib/Common.h}
	\subsection{Gestion.h}
		\lstinputlisting[language=C]{../lib/Gestion.h}
	\subsection{Gestion.c}
		\lstinputlisting[language=C]{../src/Gestion.c}
	\subsection{Boat.h}
		\lstinputlisting[language=C]{../lib/Boat.h}
	\subsection{Boat.c}
		\lstinputlisting[language=C]{../src/Boat.c}
	\subsection{Port.h}
		\lstinputlisting[language=C]{../lib/Port.h}
	\subsection{Port.c}
		\lstinputlisting[language=C]{../src/Port.c}
	\subsection{Dock.h}
		\lstinputlisting[language=C]{../lib/Dock.h}
	\subsection{Dock.c}
		\lstinputlisting[language=C]{../src/Dock.c}
	\subsection{GenVehicle.h}
		\lstinputlisting[language=C]{../lib/GenVehicle.h}
	\subsection{GenVehicle.c}
		\lstinputlisting[language=C]{../src/GenVehicle.c}

\section{Conclusion}
			%
			%	TODO
			%

\end{document}